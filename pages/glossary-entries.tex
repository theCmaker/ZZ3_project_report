\newglossaryentry{algorithme genetique}
{
	name={algorithme génétique},
	first={algorithme génétique}\textsuperscript{$\ddagger$}, % Première occurrence marquée par le double obèle.
	firstplural={algorithmes génétiques}\textsuperscript{$\ddagger$},
	plural={algorithmes génétiques},
	description={Les algorithmes génétiques appartiennent à la famille des algorithmes évolutionnistes. Leur but est d'obtenir une solution approchée à un problème d'optimisation, lorsqu'il n'existe pas de méthode exacte (ou que la solution est inconnue) pour le résoudre en un temps raisonnable. Les algorithmes génétiques utilisent la notion de sélection naturelle et l'appliquent à une population de solutions potentielles au problème donné.}%,
	%user1={Variante 1\textsuperscript{$\ddagger$}}
}

\newacronym{brkga}{BRKGA}{Biased Random-Key Genetic Algorithm}

\newglossaryentry{cache}
{
	name=cache,
	first=cache\textsuperscript{$\ddagger$}, % Première occurrence marquée par le double obèle.
	firstplural=caches\textsuperscript{$\ddagger$},
	plural=caches,
	description={Une mémoire cache ou antémémoire est, en informatique, une mémoire qui enregistre temporairement des copies de données provenant d'une source, afin de diminuer le temps d'un accès ultérieur (en lecture) d'un matériel informatique (en général, un processeur) à ces données. \cite{wikipedia-cache}}%,
	%user1={Variante 1\textsuperscript{$\ddagger$}}
}

\newglossaryentry{heuristique}
{
	name={heuristique},
	first={heuristique}\textsuperscript{$\ddagger$}, % Première occurrence marquée par le double obèle.
	firstplural={heuristiques}\textsuperscript{$\ddagger$},
	plural={heuristiques},
	description={Une heuristique est une méthode de calcul qui fournit rapidement (en temps polynomial) une solution réalisable, pas nécessairement optimale, pour un problème d'optimisation.}%,
	%user1={Variante 1\textsuperscript{$\ddagger$}}
}

\newglossaryentry{metaheuristique}
{
	name={métaheuristique},
	first={métaheuristique}\textsuperscript{$\ddagger$}, % Première occurrence marquée par le double obèle.
	firstplural={métaheuristiques}\textsuperscript{$\ddagger$},
	plural={métaheuristiques},
	description={Une métaheuristique est un algorithme d’optimisation visant à résoudre des problèmes d’optimisation difficile. Les métaheuristiques sont généralement des algorithmes stochastiques itératifs, qui progressent vers un optimum global, c'est-à-dire l'extremum global d'une fonction, par échantillonnage d’une fonction objectif. Elles se comportent comme des algorithmes de recherche, tentant d’apprendre les caractéristiques d’un problème afin d’en trouver une approximation de la meilleure solution.}%,
	%user1={Variante 1\textsuperscript{$\ddagger$}}
}

\newacronym{msels}{MS-ELS}{Multi-Start Evolutionary Local Search}

\newacronym{stdvrp}{STDVRP}{Selective Time Dependent Vehicule Routing Problem}

\newglossaryentry{tournee}
{
	name={tournée},
	first={tournée}\textsuperscript{$\ddagger$}, % Première occurrence marquée par le double obèle.
	firstplural={tournées}\textsuperscript{$\ddagger$},
	plural={tournées},
	description={Séquence de mobiles inspectés par un intercepteur.}%,
	%user1={Variante 1\textsuperscript{$\ddagger$}}
}

\newacronym{vnd}{VND}{Vertical Neighbourhood Descent}
\newacronym[longplural={Vehicule Routing Problems}]{vrp}{VRP}{Vehicule Routing Problem}
