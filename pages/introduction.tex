\chapter*{Introduction}
\addcontentsline{toc}{chapter}{Introduction}

Ce projet nous a été proposé par Christophe Duhamel, enseignant à l’ISIMA et chercheur au LIMOS.

Il s'inscrit dans la continuité d'un projet réalisé en première année, consistant en une introduction aux problèmes de tournées de véhicules.
Il s'agit d'un problème où les clients sont mobiles et se déplacent selon une trajectoire définie à une vitesse fixe. Au terme de ce premier projet, le calcul d'une seule et unique tournée était réalisable.

Les objectifs liés à ce projet sont multiples. Dans un premier temps, nous avons souhaité adapter les calculs pour permettre la construction de plusieurs tournées. Par la suite nous avons amélioré les résultats fournis par ces calculs de manière à optimiser les tournées selon deux critères: le temps nécessaire pour terminer la tournée la plus longue (en termes de durée), et le nombre total de clients livrés.

La prise en compte de ces deux critères permet d'inscrire le projet dans les conditions qualifiant des problèmes de secours (actions humanitaires) ou de protection des civils (actions militaires). En effet, dans ces deux types de problèmes, le temps est un critère déterminant dans la réactivité et le succès des interventions, et le nombre d'objectifs atteints vise à généraliser ce succès. On peut donc voir à ce projet des utilisations multiples comme l'inspection de sites sensibles, le suivi de déplacements de troupes, l'arrestation de convois de type "go-fast" ou la livraison de matériel de secours, vivres, équipements de première nécessité à des équipes ou des groupes mobiles ou immobiles.

Pour répondre à ces problématiques, nous avons mis en \oe uvre différentes solutions et méthodes de résolution que nous décrirons dans les chapitres suivants. Nous avons attaché une attention particulière à la qualité des résultats, l'optimisation des calculs et la réutilisabilité de notre travail.

Pour des raisons de performances, nos développements ont été réalisés en langage C++, mais les principes que nous évoquerons sont applicables à d'autres langages et paradigmes.