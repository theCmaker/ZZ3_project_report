\chapter{Résultats}
    \section{Avant améliorations}
        Nous avons comparé les performances de l'heuristique d'insertion au plus tôt dans les cas où elle fonctionne avec ou sans cache. Cela nous a permis de justifier de l'intérêt de ce cache, surtout sur de grosses instances. Nous avons donc mesuré les temps de calcul de l'heuristique sur différentes instances. Les résultats sont obtenus par moyennage sur 100 exécutions et sont présentés dans le tableau~\ref{tab:cache-perf}.

        \begin{table}[H]
            \centering
            \begin{tabular}{l|r|r|l}
            \cline{2-3}
                                                                     & \multicolumn{1}{c|}{\textbf{Avec Cache}} & \multicolumn{1}{c|}{\textbf{Sans Cache}} &  \\ \cline{1-3}
            \multicolumn{1}{|l|}{\textbf{instance A (10 mobiles)}}   & 32.291 µs                               & 53.106 µs                               &  \\ \cline{1-3}
            \multicolumn{1}{|l|}{\textbf{instance B (20 mobiles)}}   & 137.305 µs                               & 274.566 µs                               &  \\ \cline{1-3}
            \multicolumn{1}{|l|}{\textbf{instance C (30 mobiles)}}   & 268.857 µs                               & 760.437 µs                               &  \\ \cline{1-3}
            \multicolumn{1}{|l|}{\textbf{instance D (40 mobiles)}}   & 487.622 µs                               & 1759.860 µs                               &  \\ \cline{1-3}
            \multicolumn{1}{|l|}{\textbf{instance E (60 mobiles)}}   & 1212.790 µs                               & 6391.220 µs                               &  \\ \cline{1-3}
            \multicolumn{1}{|l|}{\textbf{instance F (100 mobiles)}}  & 3038.210µs                                & 17595.600 µs                               &  \\ \cline{1-3}
            \multicolumn{1}{|l|}{\textbf{instance G (1000 mobiles)}} & 486714.000 µs                                & 8.69975e+06 µs                           &  \\ \cline{1-3}
            \end{tabular}
            \caption{Performances du cache}
            \label{tab:cache-perf}
        \end{table}
    	Résultats depuis le passage au C++, le gain en temps avec le cache, les corrections et améliorations des heuristiques de construction.
    	Comparaison avec les résultats ZZ1.
    \section{Après améliorations}
    	\subsection{Résultats de la VND}
    	\subsection{Résultats du MS-ELS}
    	\subsection{Résultats du BRKGA}

            \begin{table}[H]
                \centering
                \begin{tabular}{r|r|r|r|r|r|l}
                \cline{2-6}
                \multicolumn{1}{l|}{\multirow{2}{*}{}}                 & \multicolumn{2}{c|}{\textbf{BRKGA -- SANS VND}}                                  & \multicolumn{3}{c|}{\textbf{BRKGA -- AVEC VND}}                                                                                &  \\ \cline{2-6}
                \multicolumn{1}{l|}{}                                  & \multicolumn{1}{c|}{\textit{temps}} & \multicolumn{1}{c|}{\textit{\# solutions}} & \multicolumn{1}{c|}{\textit{temps}} & \multicolumn{1}{c|}{\textit{\# solutions}} & \multicolumn{1}{c|}{\textit{\# appels VND}} &  \\ \cline{1-6}
                \multicolumn{1}{|l|}{\textbf{instance A (10 mobiles)}} & 1.30s                               & 10                                         & 367.37s                             & 18                                         & 247729                                      &  \\ \cline{1-6}
                \multicolumn{1}{|l|}{\textbf{instance B (20 mobiles)}} & 1.34s                               & 20                                         & 1046.95s                            & 20                                         & 336071                                      &  \\ \cline{1-6}
                \multicolumn{1}{|l|}{\textbf{instance C (30 mobiles)}} & 1.98s                               & 22                                         & 339.88s                             & 6                                          & 183920                                      &  \\ \cline{1-6}
                \multicolumn{1}{|l|}{\textbf{instance D (40 mobiles)}} & 8.93s                               & 34                                         & 5656.13s                            & 20                                         & 382957                                      &  \\ \cline{1-6}
                \end{tabular}
                \caption{Résultats et performance de BRKGA}
                \label{tab:brkga}
            \end{table}
            \subsubsection{10m2i2d}
                \begin{tikzpicture}[xscale=1.21212,yscale=0.1585229]
\draw[xstep=1.5,ystep=8,thin,dotted,color=Black] (-0.45,-1.75657) grid (9.44448,52.0675);
\begin{scope}
  \clip (-0.45,-1.75657) rectangle (9.44448,52.0675);
  \definecolor{hvColor}{RGB}{128,0,0}
  \draw[color=hvColor, fill=hvColor, fill opacity=0.4] (0,0.692884) -- (0,49.682) -- (1,19.0169) -- (2,9.42242) -- (3,7.37304) -- (4,5.26681) -- (5,3.78305) -- (6,3.43859) -- (8,0.692884) -| (9,0.692884) |- (0,0) -- cycle;
  \definecolor{pLineColor}{RGB}{128,0,0}
  \definecolor{pPointColor}{RGB}{0,40,240}
  \draw[thick,color=pPointColor] (0,49.682) node[draw,color=pPointColor,fill=pPointColor, inner sep = 0pt, minimum size=2mm] {} -- (1,19.0169) node[draw,color=pPointColor,fill=pPointColor, inner sep = 0pt, minimum size=2mm] {} -- (2,9.42242) node[draw,color=pPointColor,fill=pPointColor, inner sep = 0pt, minimum size=2mm] {} -- (3,7.37304) node[draw,color=pPointColor,fill=pPointColor, inner sep = 0pt, minimum size=2mm] {} -- (4,5.26681) node[draw,color=pPointColor,fill=pPointColor, inner sep = 0pt, minimum size=2mm] {} -- (5,3.78305) node[draw,color=pPointColor,fill=pPointColor, inner sep = 0pt, minimum size=2mm] {} -- (6,3.43859) node[draw,color=pPointColor,fill=pPointColor, inner sep = 0pt, minimum size=2mm] {} -- (8,0.692884) node[draw,color=pPointColor,fill=pPointColor, inner sep = 0pt, minimum size=2mm] {};
  \definecolor{pLineColor}{RGB}{255,16,16}
  \definecolor{pPointColor}{RGB}{255,16,16}
  \node[draw,color=pPointColor,fill=pPointColor, inner sep = 0pt, minimum size=2mm] at (7,3.43859) {};
  \node[draw,color=pPointColor,fill=pPointColor, inner sep = 0pt, minimum size=2mm] at (9,0.692884) {};
\end{scope}
\draw[->,>=triangle 45] (-0.45,-1.75657) -- coordinate (x axis mid) (9.44448,-1.75657);
\node[below=1cm,anchor=center] at (x axis mid) {nombre de mobiles ratés};
\foreach \x in {0,1.5,3,4.5,6,7.5,9}
  \draw (\x,-1.75657) -- (\x,-1.75657) node[anchor=north] {\x};
\draw[->,>=triangle 45] (-0.45,-1.75657) -- coordinate (y axis mid) (-0.45,52.0675);
\node[left=1cm,rotate=90,anchor=center] at (y axis mid) {temps de la pire tournée};
\foreach \y in {0,8,16,24,32,40,48}
  \draw (-0.45,\y) -- (-0.45,\y) node[anchor=east] {\y};
\end{tikzpicture}
                
                \begin{tikzpicture}[xscale=1.21212,yscale=0.0585229]
\draw[xstep=1.5,ystep=15,thin,dotted,color=Black] (-0.45,-2.35859) grid (9.44446,117.04);
\begin{scope}
  \clip (-0.45,-2.35859) rectangle (9.44446,117.04);
  \definecolor{hvColor}{RGB}{128,0,0}
  \draw[color=hvColor, fill=hvColor, fill opacity=0.4] (0,3.07506) -- (0,7.96854) -- (1,6.82064) -- (2,5.94912) -- (6,3.07506) -| (9,3.07506) |- (0,0) -- cycle;
  \definecolor{pLineColor}{RGB}{186,0,0}
  \definecolor{pPointColor}{RGB}{186,0,0}
  \node[draw,color=pPointColor,fill=pPointColor, inner sep = 0pt, minimum size=2mm] at (0,8.10609) {};
  \node[draw,color=pPointColor,fill=pPointColor, inner sep = 0pt, minimum size=2mm] at (1,7.01988) {};
  \node[draw,color=pPointColor,fill=pPointColor, inner sep = 0pt, minimum size=2mm] at (8,3.63866) {};
  \definecolor{pLineColor}{RGB}{244,0,0}
  \definecolor{pPointColor}{RGB}{244,0,0}
  \node[draw,color=pPointColor,fill=pPointColor, inner sep = 0pt, minimum size=2mm] at (0,13.064) {};
  \node[draw,color=pPointColor,fill=pPointColor, inner sep = 0pt, minimum size=2mm] at (6,12.1234) {};
  \node[draw,color=pPointColor,fill=pPointColor, inner sep = 0pt, minimum size=2mm] at (9,3.63866) {};
  \definecolor{pLineColor}{RGB}{255,45,45}
  \definecolor{pPointColor}{RGB}{255,45,45}
  \node[draw,color=pPointColor,fill=pPointColor, inner sep = 0pt, minimum size=2mm] at (0,111.748) {};
  \node[draw,color=pPointColor,fill=pPointColor, inner sep = 0pt, minimum size=2mm] at (1,59.0744) {};
  \node[draw,color=pPointColor,fill=pPointColor, inner sep = 0pt, minimum size=2mm] at (2,49.4226) {};
  \node[draw,color=pPointColor,fill=pPointColor, inner sep = 0pt, minimum size=2mm] at (3,24.3197) {};
  \node[draw,color=pPointColor,fill=pPointColor, inner sep = 0pt, minimum size=2mm] at (5,17.872) {};
  \node[draw,color=pPointColor,fill=pPointColor, inner sep = 0pt, minimum size=2mm] at (7,12.1234) {};
  \definecolor{pLineColor}{RGB}{255,103,103}
  \definecolor{pPointColor}{RGB}{255,103,103}
  \node[draw,color=pPointColor,fill=pPointColor, inner sep = 0pt, minimum size=2mm] at (4,24.3197) {};
  \definecolor{pLineColor}{RGB}{128,0,0}
  \definecolor{pPointColor}{RGB}{0,40,240}
  \draw[thick,color=pPointColor] (0,7.96854) node[draw,color=pPointColor,fill=pPointColor, inner sep = 0pt, minimum size=2mm] {} -- (1,6.82064) node[draw,color=pPointColor,fill=pPointColor, inner sep = 0pt, minimum size=2mm] {} -- (2,5.94912) node[draw,color=pPointColor,fill=pPointColor, inner sep = 0pt, minimum size=2mm] {} -- (6,3.07506) node[draw,color=pPointColor,fill=pPointColor, inner sep = 0pt, minimum size=2mm] {};
\end{scope}
\draw[->,>=triangle 45] (-0.45,-2.35859) -- coordinate (x axis mid) (9.44446,-2.35859);
\node[below=1cm,anchor=center] at (x axis mid) {nombre de mobiles ratés};
\foreach \x in {0,1.5,3,4.5,6,7.5,9}
  \draw (\x,-2.35859) -- (\x,-2.35859) node[anchor=north] {\x};
\draw[->,>=triangle 45] (-0.45,-2.35859) -- coordinate (y axis mid) (-0.45,117.04);
\node[left=1cm,rotate=90,anchor=center] at (y axis mid) {temps de la pire tournée};
\foreach \y in {0,15,30,45,60,75,90,105}
  \draw (-0.45,\y) -- (-0.45,\y) node[anchor=east] {\y};
\end{tikzpicture}
            \subsubsection{20m3i3d}
                \begin{tikzpicture}[xscale=0.574163,yscale=0.057214]
\draw[xstep=3,ystep=20,thin,dotted,color=Black] (-0.95,-5.85439) grid (19.9383,132.868);
\begin{scope}
  \clip (-0.95,-5.85439) rectangle (19.9383,132.868);
  \definecolor{hvColor}{RGB}{128,0,0}
  \draw[color=hvColor, fill=hvColor, fill opacity=0.4] (0,0.458678) -- (0,126.72) -- (1,47.0057) -- (2,34.308) -- (3,18.7744) -- (4,18.631) -- (5,17.191) -- (6,14.4692) -- (9,10.3724) -- (13,6.32048) -- (15,2.61757) -- (16,1.8906) -- (17,1.44591) -- (19,0.458678) -| (19,0.458678) |- (0,0) -- cycle;
  \definecolor{pLineColor}{RGB}{200,0,0}
  \definecolor{pPointColor}{RGB}{200,0,0}
  \node[draw,color=pPointColor,fill=pPointColor, inner sep = 0pt, minimum size=2mm] at (7,14.4692) {};
  \node[draw,color=pPointColor,fill=pPointColor, inner sep = 0pt, minimum size=2mm] at (10,10.3724) {};
  \node[draw,color=pPointColor,fill=pPointColor, inner sep = 0pt, minimum size=2mm] at (14,6.32048) {};
  \node[draw,color=pPointColor,fill=pPointColor, inner sep = 0pt, minimum size=2mm] at (18,1.44591) {};
  \definecolor{pLineColor}{RGB}{255,16,16}
  \definecolor{pPointColor}{RGB}{255,16,16}
  \node[draw,color=pPointColor,fill=pPointColor, inner sep = 0pt, minimum size=2mm] at (8,14.4692) {};
  \node[draw,color=pPointColor,fill=pPointColor, inner sep = 0pt, minimum size=2mm] at (11,10.3724) {};
  \definecolor{pLineColor}{RGB}{255,88,88}
  \definecolor{pPointColor}{RGB}{255,88,88}
  \node[draw,color=pPointColor,fill=pPointColor, inner sep = 0pt, minimum size=2mm] at (12,10.3724) {};
  \definecolor{pLineColor}{RGB}{128,0,0}
  \definecolor{pPointColor}{RGB}{0,40,240}
  \draw[thick,color=pPointColor] (0,126.72) node[draw,color=pPointColor,fill=pPointColor, inner sep = 0pt, minimum size=2mm] {} -- (1,47.0057) node[draw,color=pPointColor,fill=pPointColor, inner sep = 0pt, minimum size=2mm] {} -- (2,34.308) node[draw,color=pPointColor,fill=pPointColor, inner sep = 0pt, minimum size=2mm] {} -- (3,18.7744) node[draw,color=pPointColor,fill=pPointColor, inner sep = 0pt, minimum size=2mm] {} -- (4,18.631) node[draw,color=pPointColor,fill=pPointColor, inner sep = 0pt, minimum size=2mm] {} -- (5,17.191) node[draw,color=pPointColor,fill=pPointColor, inner sep = 0pt, minimum size=2mm] {} -- (6,14.4692) node[draw,color=pPointColor,fill=pPointColor, inner sep = 0pt, minimum size=2mm] {} -- (9,10.3724) node[draw,color=pPointColor,fill=pPointColor, inner sep = 0pt, minimum size=2mm] {} -- (13,6.32048) node[draw,color=pPointColor,fill=pPointColor, inner sep = 0pt, minimum size=2mm] {} -- (15,2.61757) node[draw,color=pPointColor,fill=pPointColor, inner sep = 0pt, minimum size=2mm] {} -- (16,1.8906) node[draw,color=pPointColor,fill=pPointColor, inner sep = 0pt, minimum size=2mm] {} -- (17,1.44591) node[draw,color=pPointColor,fill=pPointColor, inner sep = 0pt, minimum size=2mm] {} -- (19,0.458678) node[draw,color=pPointColor,fill=pPointColor, inner sep = 0pt, minimum size=2mm] {};
\end{scope}
\draw[->,>=triangle 45] (-0.95,-5.85439) -- coordinate (x axis mid) (19.9383,-5.85439);
\node[below=1cm,anchor=center] at (x axis mid) {nombre de mobiles ratés};
\foreach \x in {0,3,6,9,12,15,18}
  \draw (\x,-5.85439) -- (\x,-5.85439) node[anchor=north] {\x};
\draw[->,>=triangle 45] (-0.95,-5.85439) -- coordinate (y axis mid) (-0.95,132.868);
\node[left=1cm,rotate=90,anchor=center] at (y axis mid) {temps de la pire tournée};
\foreach \y in {0,20,40,60,80,100,120}
  \draw (-0.95,\y) -- (-0.95,\y) node[anchor=east] {\y};
\end{tikzpicture}
                
                \begin{tikzpicture}[xscale=0.574163,yscale=0.0337214]
\draw[xstep=3,ystep=35,thin,dotted,color=Black] (-0.95,-7.89917) grid (19.9383,211.55);
\begin{scope}
  \clip (-0.95,-7.89917) rectangle (19.9383,211.55);
  \definecolor{hvColor}{RGB}{128,0,0}
  \draw[color=hvColor, fill=hvColor, fill opacity=0.4] (0,2.08765) -- (0,10.4594) -- (4,7.97055) -- (6,6.71603) -- (15,6.55359) -- (16,5.14916) -- (18,2.08765) -| (19,2.08765) |- (0,0) -- cycle;

  \definecolor{pLineColor}{RGB}{160,0,0}
  \definecolor{pPointColor}{RGB}{160,0,0}
  \node [draw,color=pPointColor,fill=pPointColor, inner sep = 0pt, minimum size=2mm] at (0,12.2265) {};
  \node [draw,color=pPointColor,fill=pPointColor, inner sep = 0pt, minimum size=2mm] at (8,7.51176) {};
  \node [draw,color=pPointColor,fill=pPointColor, inner sep = 0pt, minimum size=2mm] at (17,5.14916) {};
  \node [draw,color=pPointColor,fill=pPointColor, inner sep = 0pt, minimum size=2mm] at (19,2.08765) {};
  \definecolor{pLineColor}{RGB}{192,0,0}
  \definecolor{pPointColor}{RGB}{192,0,0}
  \node [draw,color=pPointColor,fill=pPointColor, inner sep = 0pt, minimum size=2mm] at (0,21.0649) {};
  \node [draw,color=pPointColor,fill=pPointColor, inner sep = 0pt, minimum size=2mm] at (8,19.2932) {};
  \node [draw,color=pPointColor,fill=pPointColor, inner sep = 0pt, minimum size=2mm] at (11,14.3302) {};
  \node [draw,color=pPointColor,fill=pPointColor, inner sep = 0pt, minimum size=2mm] at (12,11.2603) {};
  \node [draw,color=pPointColor,fill=pPointColor, inner sep = 0pt, minimum size=2mm] at (18,5.14916) {};
  \definecolor{pLineColor}{RGB}{224,0,0}
  \definecolor{pPointColor}{RGB}{224,0,0}
  \node [draw,color=pPointColor,fill=pPointColor, inner sep = 0pt, minimum size=2mm] at (0,22.1747) {};
  \node [draw,color=pPointColor,fill=pPointColor, inner sep = 0pt, minimum size=2mm] at (7,21.5963) {};
  \node [draw,color=pPointColor,fill=pPointColor, inner sep = 0pt, minimum size=2mm] at (9,19.2932) {};
  \node [draw,color=pPointColor,fill=pPointColor, inner sep = 0pt, minimum size=2mm] at (13,11.2603) {};
  \definecolor{pLineColor}{RGB}{255,1,1}
  \definecolor{pPointColor}{RGB}{255,1,1}
  \node [draw,color=pPointColor,fill=pPointColor, inner sep = 0pt, minimum size=2mm] at (0,22.4093) {};
  \node [draw,color=pPointColor,fill=pPointColor, inner sep = 0pt, minimum size=2mm] at (10,19.2932) {};
  \node [draw,color=pPointColor,fill=pPointColor, inner sep = 0pt, minimum size=2mm] at (14,11.2603) {};
  \definecolor{pLineColor}{RGB}{255,33,33}
  \definecolor{pPointColor}{RGB}{255,33,33}
  \node [draw,color=pPointColor,fill=pPointColor, inner sep = 0pt, minimum size=2mm] at (0,23.4044) {};
  \definecolor{pLineColor}{RGB}{255,65,65}
  \definecolor{pPointColor}{RGB}{255,65,65}
  \node [draw,color=pPointColor,fill=pPointColor, inner sep = 0pt, minimum size=2mm] at (0,201.824) {};
  \node [draw,color=pPointColor,fill=pPointColor, inner sep = 0pt, minimum size=2mm] at (1,67.1062) {};
  \node [draw,color=pPointColor,fill=pPointColor, inner sep = 0pt, minimum size=2mm] at (3,36.5789) {};
  \node [draw,color=pPointColor,fill=pPointColor, inner sep = 0pt, minimum size=2mm] at (6,33.6687) {};
  \definecolor{pLineColor}{RGB}{255,97,97}
  \definecolor{pPointColor}{RGB}{255,97,97}
  \node [draw,color=pPointColor,fill=pPointColor, inner sep = 0pt, minimum size=2mm] at (2,67.1062) {};
  \node [draw,color=pPointColor,fill=pPointColor, inner sep = 0pt, minimum size=2mm] at (4,36.5789) {};
  \definecolor{pLineColor}{RGB}{255,129,129}
  \definecolor{pPointColor}{RGB}{255,129,129}
  \node [draw,color=pPointColor,fill=pPointColor, inner sep = 0pt, minimum size=2mm] at (5,36.5789) {};
  \definecolor{pLineColor}{RGB}{128,0,0}
  \definecolor{pPointColor}{RGB}{0,40,240}
  \draw[thick,color=pPointColor] (0,10.4594) node[draw,color=pPointColor,fill=pPointColor, inner sep = 0pt, minimum size=2mm] {} -- (4,7.97055) node[draw,color=pPointColor,fill=pPointColor, inner sep = 0pt, minimum size=2mm] {} -- (6,6.71603) node[draw,color=pPointColor,fill=pPointColor, inner sep = 0pt, minimum size=2mm] {} -- (15,6.55359) node[draw,color=pPointColor,fill=pPointColor, inner sep = 0pt, minimum size=2mm] {} -- (16,5.14916) node[draw,color=pPointColor,fill=pPointColor, inner sep = 0pt, minimum size=2mm] {} -- (18,2.08765) node[draw,color=pPointColor,fill=pPointColor, inner sep = 0pt, minimum size=2mm] {};
\end{scope}
\draw[->,>=triangle 45] (-0.95,-7.89917) -- coordinate (x axis mid) (19.9383,-7.89917);
\node[below=1cm,anchor=center] at (x axis mid) {nombre de mobiles ratés};
\foreach \x in {0,3,6,9,12,15,18}
  \draw (\x,-7.89917) -- (\x,-7.89917) node[anchor=north] {\x};
\draw[->,>=triangle 45] (-0.95,-7.89917) -- coordinate (y axis mid) (-0.95,211.55);
\node[left=1cm,rotate=90,anchor=center] at (y axis mid) {temps de la pire tournée};
\foreach \y in {0,35,70,105,140,175,210}
  \draw (-0.95,\y) -- (-0.95,\y) node[anchor=east] {\y};
\end{tikzpicture}
            \subsubsection{30m4i3d}
                \begin{tikzpicture}[xscale=0.519481,yscale=0.278316]
\draw[xstep=3.5,ystep=4,thin,dotted,color=Black] (6.95,1.59268) grid (30.0351,28.2723);
\begin{scope}
  \clip (6.95,1.59268) rectangle (30.0351,28.2723);
  \definecolor{hvColor}{RGB}{128,0,0}
  \draw[color=hvColor, fill=hvColor, fill opacity=0.4] (8,2.80689) -- (8,27.091) -- (9,18.995) -- (10,10.6267) -- (11,9.26623) -- (12,9.24041) -- (13,6.41101) -- (16,5.58238) -- (17,4.13268) -- (19,3.39565) -- (24,2.80689) -| (29,2.80689) -- cycle;
  \definecolor{pLineColor}{RGB}{128,0,0}
  \definecolor{pPointColor}{RGB}{0,40,240}
  \draw[thick,color=pPointColor] (8,27.091) node[draw,color=pPointColor,fill=pPointColor, inner sep = 0pt, minimum size=2mm] {} -- (9,18.995) node[draw,color=pPointColor,fill=pPointColor, inner sep = 0pt, minimum size=2mm] {} -- (10,10.6267) node[draw,color=pPointColor,fill=pPointColor, inner sep = 0pt, minimum size=2mm] {} -- (11,9.26623) node[draw,color=pPointColor,fill=pPointColor, inner sep = 0pt, minimum size=2mm] {} -- (12,9.24041) node[draw,color=pPointColor,fill=pPointColor, inner sep = 0pt, minimum size=2mm] {} -- (13,6.41101) node[draw,color=pPointColor,fill=pPointColor, inner sep = 0pt, minimum size=2mm] {} -- (16,5.58238) node[draw,color=pPointColor,fill=pPointColor, inner sep = 0pt, minimum size=2mm] {} -- (17,4.13268) node[draw,color=pPointColor,fill=pPointColor, inner sep = 0pt, minimum size=2mm] {} -- (19,3.39565) node[draw,color=pPointColor,fill=pPointColor, inner sep = 0pt, minimum size=2mm] {} -- (24,2.80689) node[draw,color=pPointColor,fill=pPointColor, inner sep = 0pt, minimum size=2mm] {};
  \definecolor{pLineColor}{RGB}{176,0,0}
  \definecolor{pPointColor}{RGB}{176,0,0}
  \draw[thick,color=pPointColor] (14,6.41101) node[draw,color=pPointColor,fill=pPointColor, inner sep = 0pt, minimum size=2mm] {} -- (18,4.13268) node[draw,color=pPointColor,fill=pPointColor, inner sep = 0pt, minimum size=2mm] {} -- (20,3.39565) node[draw,color=pPointColor,fill=pPointColor, inner sep = 0pt, minimum size=2mm] {} -- (25,2.80689) node[draw,color=pPointColor,fill=pPointColor, inner sep = 0pt, minimum size=2mm] {};
  \definecolor{pLineColor}{RGB}{224,0,0}
  \definecolor{pPointColor}{RGB}{224,0,0}
  \draw[thick,color=pPointColor] (15,6.41101) node[draw,color=pPointColor,fill=pPointColor, inner sep = 0pt, minimum size=2mm] {} -- (21,3.39565) node[draw,color=pPointColor,fill=pPointColor, inner sep = 0pt, minimum size=2mm] {} -- (26,2.80689) node[draw,color=pPointColor,fill=pPointColor, inner sep = 0pt, minimum size=2mm] {};
  \definecolor{pLineColor}{RGB}{255,16,16}
  \definecolor{pPointColor}{RGB}{255,16,16}
  \draw[thick,color=pPointColor] (22,3.39565) node[draw,color=pPointColor,fill=pPointColor, inner sep = 0pt, minimum size=2mm] {} -- (27,2.80689) node[draw,color=pPointColor,fill=pPointColor, inner sep = 0pt, minimum size=2mm] {};
  \definecolor{pLineColor}{RGB}{255,65,65}
  \definecolor{pPointColor}{RGB}{255,65,65}
  \draw[thick,color=pPointColor] (23,3.39565) node[draw,color=pPointColor,fill=pPointColor, inner sep = 0pt, minimum size=2mm] {} -- (28,2.80689) node[draw,color=pPointColor,fill=pPointColor, inner sep = 0pt, minimum size=2mm] {};
  \definecolor{pLineColor}{RGB}{255,112,112}
  \definecolor{pPointColor}{RGB}{255,112,112}
  \draw[thick,color=pPointColor] (29,2.80689) node[draw,color=pPointColor,fill=pPointColor, inner sep = 0pt, minimum size=2mm] {};
\end{scope}
\draw[->,>=triangle 45] (6.95,1.59268) -- coordinate (x axis mid) (30.0351,1.59268);
\node[below=1cm,anchor=center] at (x axis mid) {nombre de mobiles ratés};
\foreach \x in {7,10.5,14,17.5,21,24.5,28}
  \draw (\x,1.59268) -- (\x,1.59268) node[anchor=north] {\x};
\draw[->,>=triangle 45] (6.95,1.59268) -- coordinate (y axis mid) (6.95,28.2723);
\node[left=1cm,rotate=90,anchor=center] at (y axis mid) {temps de la pire tournée};
\foreach \y in {4,8,12,16,20,24,28}
  \draw (6.95,\y) -- (6.95,\y) node[anchor=east] {\y};
\end{tikzpicture}
                
                \begin{tikzpicture}[xscale=0.727273,yscale=0.389642]
\draw[xstep=2.5,ystep=2.5,thin,dotted,color=Black] (11.25,5.89481) grid (27.7393,22.3861);
\begin{scope}
  \clip (11.25,5.89481) rectangle (27.7393,22.3861);
  \definecolor{hvColor}{RGB}{128,0,0}
  \draw[color=hvColor, fill=hvColor, fill opacity=0.4] (12,6.64534) -- (12,6.64534) -| (27,6.64534) -- cycle;
  \definecolor{pLineColor}{RGB}{128,0,0}
  \definecolor{pPointColor}{RGB}{0,40,240}
  \draw[thick,color=pPointColor] (12,6.64534) node[draw,color=pPointColor,fill=pPointColor, inner sep = 0pt, minimum size=2mm] {};
  \definecolor{pLineColor}{RGB}{224,0,0}
  \definecolor{pPointColor}{RGB}{224,0,0}
  \draw[thick,color=pPointColor] (12,21.6559) node[draw,color=pPointColor,fill=pPointColor, inner sep = 0pt, minimum size=2mm] {} -- (13,14.5307) node[draw,color=pPointColor,fill=pPointColor, inner sep = 0pt, minimum size=2mm] {} -- (17,7.13526) node[draw,color=pPointColor,fill=pPointColor, inner sep = 0pt, minimum size=2mm] {};
  \definecolor{pLineColor}{RGB}{255,65,65}
  \definecolor{pPointColor}{RGB}{255,65,65}
  \draw[thick,color=pPointColor] (22,15.147) node[draw,color=pPointColor,fill=pPointColor, inner sep = 0pt, minimum size=2mm] {} -- (27,7.74618) node[draw,color=pPointColor,fill=pPointColor, inner sep = 0pt, minimum size=2mm] {};
\end{scope}
\draw[->,>=triangle 45] (11.25,5.89481) -- coordinate (x axis mid) (27.7393,5.89481);
\node[below=1cm,anchor=center] at (x axis mid) {nombre de mobiles ratés};
\foreach \x in {12.5,15,17.5,20,22.5,25,27.5}
  \draw (\x,5.89481) -- (\x,5.89481) node[anchor=north] {\x};
\draw[->,>=triangle 45] (11.25,5.89481) -- coordinate (y axis mid) (11.25,22.3861);
\node[left=1cm,rotate=90,anchor=center] at (y axis mid) {temps de la pire tournée};
\foreach \y in {7.5,10,12.5,15,17.5,20}
  \draw (11.25,\y) -- (11.25,\y) node[anchor=east] {\y};
\end{tikzpicture}
            \subsubsection{40m5i4d}
                \begin{tikzpicture}[xscale=0.330579,yscale=0.17711]
\draw[xstep=6,ystep=4,thin,dotted,color=Black] (4.35,6.10684) grid (40.6267,40.4468);
\begin{scope}
  \clip (4.35,6.10684) rectangle (40.6267,40.4468);
  \definecolor{hvColor}{RGB}{128,0,0}
  \draw[color=hvColor, fill=hvColor, fill opacity=0.4] (6,7.66967) -- (6,38.9263) -- (7,25.7658) -- (8,21.6293) -- (10,16.8574) -- (13,14.3903) -- (14,12.4345) -- (15,11.91) -- (18,10.2815) -- (19,9.10052) -- (20,8.28724) -- (21,7.66967) -| (39,7.66967) -- cycle;
  \definecolor{pLineColor}{RGB}{128,0,0}
  \definecolor{pPointColor}{RGB}{0,40,240}
  \draw[thick,color=pPointColor] (6,38.9263) node[draw,color=pPointColor,fill=pPointColor, inner sep = 0pt, minimum size=2mm] {} -- (7,25.7658) node[draw,color=pPointColor,fill=pPointColor, inner sep = 0pt, minimum size=2mm] {} -- (8,21.6293) node[draw,color=pPointColor,fill=pPointColor, inner sep = 0pt, minimum size=2mm] {} -- (10,16.8574) node[draw,color=pPointColor,fill=pPointColor, inner sep = 0pt, minimum size=2mm] {} -- (13,14.3903) node[draw,color=pPointColor,fill=pPointColor, inner sep = 0pt, minimum size=2mm] {} -- (14,12.4345) node[draw,color=pPointColor,fill=pPointColor, inner sep = 0pt, minimum size=2mm] {} -- (15,11.91) node[draw,color=pPointColor,fill=pPointColor, inner sep = 0pt, minimum size=2mm] {} -- (18,10.2815) node[draw,color=pPointColor,fill=pPointColor, inner sep = 0pt, minimum size=2mm] {} -- (19,9.10052) node[draw,color=pPointColor,fill=pPointColor, inner sep = 0pt, minimum size=2mm] {} -- (20,8.28724) node[draw,color=pPointColor,fill=pPointColor, inner sep = 0pt, minimum size=2mm] {} -- (21,7.66967) node[draw,color=pPointColor,fill=pPointColor, inner sep = 0pt, minimum size=2mm] {};
  \definecolor{pLineColor}{RGB}{142,0,0}
  \definecolor{pPointColor}{RGB}{142,0,0}
  \draw[thick,color=pPointColor] (9,21.6293) node[draw,color=pPointColor,fill=pPointColor, inner sep = 0pt, minimum size=2mm] {} -- (11,16.8574) node[draw,color=pPointColor,fill=pPointColor, inner sep = 0pt, minimum size=2mm] {} -- (16,11.91) node[draw,color=pPointColor,fill=pPointColor, inner sep = 0pt, minimum size=2mm] {} -- (22,7.66967) node[draw,color=pPointColor,fill=pPointColor, inner sep = 0pt, minimum size=2mm] {};
  \definecolor{pLineColor}{RGB}{158,0,0}
  \definecolor{pPointColor}{RGB}{158,0,0}
  \draw[thick,color=pPointColor] (12,16.8574) node[draw,color=pPointColor,fill=pPointColor, inner sep = 0pt, minimum size=2mm] {} -- (17,11.91) node[draw,color=pPointColor,fill=pPointColor, inner sep = 0pt, minimum size=2mm] {} -- (23,7.66967) node[draw,color=pPointColor,fill=pPointColor, inner sep = 0pt, minimum size=2mm] {};
  \definecolor{pLineColor}{RGB}{173,0,0}
  \definecolor{pPointColor}{RGB}{173,0,0}
  \draw[thick,color=pPointColor] (24,7.66967) node[draw,color=pPointColor,fill=pPointColor, inner sep = 0pt, minimum size=2mm] {};
  \definecolor{pLineColor}{RGB}{188,0,0}
  \definecolor{pPointColor}{RGB}{188,0,0}
  \draw[thick,color=pPointColor] (25,7.66967) node[draw,color=pPointColor,fill=pPointColor, inner sep = 0pt, minimum size=2mm] {};
  \definecolor{pLineColor}{RGB}{204,0,0}
  \definecolor{pPointColor}{RGB}{204,0,0}
  \draw[thick,color=pPointColor] (26,7.66967) node[draw,color=pPointColor,fill=pPointColor, inner sep = 0pt, minimum size=2mm] {};
  \definecolor{pLineColor}{RGB}{219,0,0}
  \definecolor{pPointColor}{RGB}{128,0,0}
  \draw[thick,color=pPointColor] (27,7.66967) node[draw,color=pPointColor,fill=pPointColor, inner sep = 0pt, minimum size=2mm] {};
  \definecolor{pLineColor}{RGB}{233,0,0}
  \definecolor{pPointColor}{RGB}{233,0,0}
  \draw[thick,color=pPointColor] (28,7.66967) node[draw,color=pPointColor,fill=pPointColor, inner sep = 0pt, minimum size=2mm] {};
  \definecolor{pLineColor}{RGB}{249,0,0}
  \definecolor{pPointColor}{RGB}{249,0,0}
  \draw[thick,color=pPointColor] (29,7.66967) node[draw,color=pPointColor,fill=pPointColor, inner sep = 0pt, minimum size=2mm] {};
  \definecolor{pLineColor}{RGB}{255,8,8}
  \definecolor{pPointColor}{RGB}{255,8,8}
  \draw[thick,color=pPointColor] (30,7.66967) node[draw,color=pPointColor,fill=pPointColor, inner sep = 0pt, minimum size=2mm] {};
  \definecolor{pLineColor}{RGB}{255,24,24}
  \definecolor{pPointColor}{RGB}{255,24,24}
  \draw[thick,color=pPointColor] (31,7.66967) node[draw,color=pPointColor,fill=pPointColor, inner sep = 0pt, minimum size=2mm] {};
  \definecolor{pLineColor}{RGB}{255,39,39}
  \definecolor{pPointColor}{RGB}{255,39,39}
  \draw[thick,color=pPointColor] (32,7.66967) node[draw,color=pPointColor,fill=pPointColor, inner sep = 0pt, minimum size=2mm] {};
  \definecolor{pLineColor}{RGB}{255,54,54}
  \definecolor{pPointColor}{RGB}{255,54,54}
  \draw[thick,color=pPointColor] (33,7.66967) node[draw,color=pPointColor,fill=pPointColor, inner sep = 0pt, minimum size=2mm] {};
  \definecolor{pLineColor}{RGB}{255,69,69}
  \definecolor{pPointColor}{RGB}{255,69,69}
  \draw[thick,color=pPointColor] (34,7.66967) node[draw,color=pPointColor,fill=pPointColor, inner sep = 0pt, minimum size=2mm] {};
  \definecolor{pLineColor}{RGB}{255,84,84}
  \definecolor{pPointColor}{RGB}{255,84,84}
  \draw[thick,color=pPointColor] (35,7.66967) node[draw,color=pPointColor,fill=pPointColor, inner sep = 0pt, minimum size=2mm] {};
  \definecolor{pLineColor}{RGB}{255,99,99}
  \definecolor{pPointColor}{RGB}{255,99,99}
  \draw[thick,color=pPointColor] (36,7.66967) node[draw,color=pPointColor,fill=pPointColor, inner sep = 0pt, minimum size=2mm] {};
  \definecolor{pLineColor}{RGB}{255,115,115}
  \definecolor{pPointColor}{RGB}{255,115,115}
  \draw[thick,color=pPointColor] (37,7.66967) node[draw,color=pPointColor,fill=pPointColor, inner sep = 0pt, minimum size=2mm] {};
  \definecolor{pLineColor}{RGB}{255,130,130}
  \definecolor{pPointColor}{RGB}{255,130,130}
  \draw[thick,color=pPointColor] (38,7.66967) node[draw,color=pPointColor,fill=pPointColor, inner sep = 0pt, minimum size=2mm] {};
  \definecolor{pLineColor}{RGB}{255,146,146}
  \definecolor{pPointColor}{RGB}{255,146,146}
  \draw[thick,color=pPointColor] (39,7.66967) node[draw,color=pPointColor,fill=pPointColor, inner sep = 0pt, minimum size=2mm] {};
\end{scope}
\draw[->,>=triangle 45] (4.35,6.10684) -- coordinate (x axis mid) (40.6267,6.10684);
\node[below=1cm,anchor=center] at (x axis mid) {nombre de mobiles ratés};
\foreach \x in {6,12,18,24,30,36}
  \draw (\x,6.10684) -- (\x,6.10684) node[anchor=north] {\x};
\draw[->,>=triangle 45] (4.35,6.10684) -- coordinate (y axis mid) (4.35,40.4468);
\node[left=1cm,rotate=90,anchor=center] at (y axis mid) {temps de la pire tournée};
\foreach \y in {8,12,16,20,24,28,32,36,40}
  \draw (4.35,\y) -- (4.35,\y) node[anchor=east] {\y};
\end{tikzpicture}
                
                \begin{tikzpicture}[xscale=0.330579,yscale=0.17711]
\draw[xstep=6,ystep=6,thin,dotted,color=Black] (4.35,4.6401) grid (40.6267,45.2371);
\begin{scope}
  \clip (4.35,4.6401) rectangle (40.6267,45.2371);
  \definecolor{hvColor}{RGB}{128,0,0}
  \draw[color=hvColor, fill=hvColor, fill opacity=0.4] (6,6.4877) -- (6,14.8468) -- (7,14.2892) -- (11,10.3245) -- (12,9.48769) -- (16,8.48855) -- (25,6.70582) -- (37,6.4877) -| (39,6.4877) -- cycle;
  \definecolor{pLineColor}{RGB}{128,0,0}
  \definecolor{pPointColor}{RGB}{0,40,240}
  \draw[thick,color=pPointColor] (6,14.8468) node[draw,color=pPointColor,fill=pPointColor, inner sep = 0pt, minimum size=2mm] {} -- (7,14.2892) node[draw,color=pPointColor,fill=pPointColor, inner sep = 0pt, minimum size=2mm] {} -- (11,10.3245) node[draw,color=pPointColor,fill=pPointColor, inner sep = 0pt, minimum size=2mm] {} -- (12,9.48769) node[draw,color=pPointColor,fill=pPointColor, inner sep = 0pt, minimum size=2mm] {} -- (16,8.48855) node[draw,color=pPointColor,fill=pPointColor, inner sep = 0pt, minimum size=2mm] {} -- (25,6.70582) node[draw,color=pPointColor,fill=pPointColor, inner sep = 0pt, minimum size=2mm] {} -- (37,6.4877) node[draw,color=pPointColor,fill=pPointColor, inner sep = 0pt, minimum size=2mm] {};
  \definecolor{pLineColor}{RGB}{224,0,0}
  \definecolor{pPointColor}{RGB}{224,0,0}
  \draw[thick,color=pPointColor] (6,28.9739) node[draw,color=pPointColor,fill=pPointColor, inner sep = 0pt, minimum size=2mm] {} -- (11,19.0276) node[draw,color=pPointColor,fill=pPointColor, inner sep = 0pt, minimum size=2mm] {} -- (18,17.3831) node[draw,color=pPointColor,fill=pPointColor, inner sep = 0pt, minimum size=2mm] {} -- (20,9.57881) node[draw,color=pPointColor,fill=pPointColor, inner sep = 0pt, minimum size=2mm] {} -- (27,7.38144) node[draw,color=pPointColor,fill=pPointColor, inner sep = 0pt, minimum size=2mm] {} -- (39,6.4877) node[draw,color=pPointColor,fill=pPointColor, inner sep = 0pt, minimum size=2mm] {};
  \definecolor{pLineColor}{RGB}{255,65,65}
  \definecolor{pPointColor}{RGB}{255,65,65}
  \draw[thick,color=pPointColor] (6,43.4396) node[draw,color=pPointColor,fill=pPointColor, inner sep = 0pt, minimum size=2mm] {} -- (7,37.1319) node[draw,color=pPointColor,fill=pPointColor, inner sep = 0pt, minimum size=2mm] {} -- (10,32.1803) node[draw,color=pPointColor,fill=pPointColor, inner sep = 0pt, minimum size=2mm] {} -- (13,27.7482) node[draw,color=pPointColor,fill=pPointColor, inner sep = 0pt, minimum size=2mm] {} -- (14,23.3577) node[draw,color=pPointColor,fill=pPointColor, inner sep = 0pt, minimum size=2mm] {} -- (16,22.8144) node[draw,color=pPointColor,fill=pPointColor, inner sep = 0pt, minimum size=2mm] {} -- (25,14.3196) node[draw,color=pPointColor,fill=pPointColor, inner sep = 0pt, minimum size=2mm] {};
\end{scope}
\draw[->,>=triangle 45] (4.35,4.6401) -- coordinate (x axis mid) (40.6267,4.6401);
\node[below=1cm,anchor=center] at (x axis mid) {nombre de mobiles ratés};
\foreach \x in {6,12,18,24,30,36}
  \draw (\x,4.6401) -- (\x,4.6401) node[anchor=north] {\x};
\draw[->,>=triangle 45] (4.35,4.6401) -- coordinate (y axis mid) (4.35,45.2371);
\node[left=1cm,rotate=90,anchor=center] at (y axis mid) {temps de la pire tournée};
\foreach \y in {6,12,18,24,30,36,42}
  \draw (4.35,\y) -- (4.35,\y) node[anchor=east] {\y};
\end{tikzpicture}


    \section{Perspectives}
    
    	\begin{itemize}
    		\item amélioration du MS-ELS (critère de sélection des meilleures solutions -> il favorise pour l'instant le temps et donne des résultats moins bons + problème du mono critère)
    		\item ajout de l'incertitude (loi de Poisson ou loin exponentielle négative ?)
    		\item multi-thread des métaheuristiques pour augmenter les performances de calcul
    	\end{itemize}
    		
