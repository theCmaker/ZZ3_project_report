\chapter*{Conclusion}
\addcontentsline{toc}{chapter}{Conclusion}

La conception et l'optimisation d'algorithmes pour résoudre des problèmes NP-Difficiles tels que le \acrlong{vrp} sont des pratiques courantes d'aide à la décision. Pour ce projet, nous étions chargés de concevoir un solveur pour un \acrlong{stdvrp} à l'aide de méthodes itératives et en reprenant un projet existant. Il consistait à atteindre le maximum d'objets mobiles dans un espace plat à l'aide d'un ou plusieurs intercepteurs. Ce problème s'inscrit dans un objectif bi-critère où l'on cherche autant à intercepter le plus grand nombre de mobiles qu'à réduire au maximum le coût en temps.

Pour cela, les heuristiques de constructions déjà présentes ont été traduites en C++, intégrées dans un modèle objet et optimisée pour obtenir des résultats plus performants. Puis une recherche locale, la VND, constituée de huit mouvements au sein d'une ou plusieurs tournées a été intégrée dans le but d'améliorer les solutions générées. Deux métaheuristiques ont ensuite été ajoutées pour perfectionner les résultats : la \acrlong{msels} et un algorithme génétique, le \acrlong{brkga}.

Les résultats obtenus sont encourageants, notamment ceux du BRKGA, car ils montrent qu'il est possible de calculer dans un temps raisonnable de bonnes solutions, permettant à la fois de rater peu de mobiles et d'optimiser le temps d'interception de la pire tournée.

Ainsi, il serait intéressant de poursuivre ce projet en y apportant quelques corrections pour améliorer encore plus les résultats et le temps de calcul. Avec une meilleure agrégation des deux critères du problème, la MS-ELS pourrait donner de meilleures solutions. L'ajout de nouvelles métaheuristiques ainsi que le passage à un programme multi-threadé permettraient aussi d'obtenir un pannel plus large pour la recherche de solutions optimales dans un temps polynomial.
