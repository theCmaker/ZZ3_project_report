\section*{Résumé}
\addcontentsline{toc}{chapter}{Résumé}

\TODO{Resumé}

\emph{Mots-clés: }

\section*{Abstract}
\addcontentsline{toc}{chapter}{Abstract}

Mobile objects identification is a common vehicle routing optimisation problem with several applications. For instance, it can be used to model and organise safety or humanitarian operations. In both cases, time may be critical and solutions must maximise the amount of visited mobiles and minimise the total needed time. This problem needs bi-criteria optimisation and is called an open vehicle routing problem with time dependency. To solve this problem, we implemented two route-building heuristics to generate feasible but not optimal solutions and a memory cache strategy was designed to speed-up the heuristics. Then, local search procedures were coded to look for improvements on the solutions and a Vertical Neighbourhood Descent was built to combine these procedures. After that, two metaheuristics based on a genetic algorithm with random keys and multi-start were used to generate better solutions on both time and amount criteria. The developed algorithms must be able to provide solutions quickly, so they were written using the C++ language and benchmarked with real-size problems. We present results of experiments on self-generated and well-known datasets. Results showed that the memory cache strategy divided the heuristic computation by ten. Several graphical representations for temporal and spatial dimensions were also provided along with a benchmark and Pareto frontiers charts to compare and rank the generated solutions. Efficiency and convergence charts justified the strategic decisions taken in the Vertical Neighbourhood Descent building process. The results obtained at the end of this project confirm that routing problems can be solved using constructive and iterative methods because of the large amount of possibilities on which an exact solver would fail or need an extremely long time to end. These results made it possible for us to validate a proof of concept in predicting how much time is needed to arrest drug smugglers on "go-fast" races.

\emph{Keywords: Decision support, Vehicle Routing Problem, Metaheuristic, Vertical Neighbourhood Descent, Bi-criteria optimisation.}