\section*{Résumé}
\addcontentsline{toc}{chapter}{Résumé}

L'identification d'objets mobiles est un problème d'optimisation courant pour les \glspl{tournee} de véhicules, doté de nombreuses applications. Par exemple, on peut l'utiliser pour modéliser et organiser des interventions de sécurité ou des opérations humanitaires. Dans les deux cas, le temps est une donnée critique et les solutions doivent maximiser le nombre de mobiles interceptés et minimiser le temps requis total. Ce problème nécessite donc une optimisation bi-critères et est appelé problème sélectif de tournées de véhicules avec dépendance de temps. Pour résoudre ce problème, nous avons implémenté deux \glspl{heuristique} de construction pour générer des solutions réalisables, mais non optimales. Une politique de cache de données a été réalisée pour accélérer les heuristiques. Ensuite, des procédures de recherche locale ont été ajoutées pour déterminer les améliorations possibles d'une solution et une Vertical Neighbouhood Descent (VND) a été construite pour combiner ces méthodes. Puis deux \glspl{metaheuristique} fondées sur un algorithme génétique à clés aléatoires et à multiples démarrages ont été utilisées pour générer de meilleures solutions sur les critères de temps et de quantité. Les algorithmes développés doivent être capables de fournir des solutions rapidement, ils ont donc été écrits dans le langage C++ et ont été calibrés avec des problèmes de taille réelle. Nous présentons les résultats de jeux de données connus et générés automatiquement. Les résultats montrent que la politique de cache de données a divisé le temps de calcul des heuristiques par dix. Plusieurs graphiques représentant les données dans l'espace et le temps sont fournis ainsi que le calibrage et la représentation des fronts de Pareto pour la comparaison et le classement des solutions générées. L'efficacité et la convergence des graphiques justifie les décisions stratégiques prises dans le processus de construction de la VND. Les résultats obtenus à la fin du projet confirment que les problèmes de tournées peuvent être résolus en utilisant des méthodes itératives de construction là où une résolution exacte échouerait ou mettrait énormément de temps à cause du très grand nombre de possibilités existantes. Ces résultats nous ont permis de valider une preuve de concept pour la prédiction du temps requis pour arrêter des trafiquants de drogues à bord de véhicules à grande vitesse.

\emph{Mots-clés: Aide à la décision, problème de tournées de véhicules, métaheuristique, Vertical Neighbourhood Descent, optimisation bi-critère.}

\newpage
\section*{Abstract}
\addcontentsline{toc}{chapter}{Abstract}

Mobile objects identification is a common vehicle routing optimisation problem with several applications. For instance, it can be used to model and organise safety or humanitarian operations. In both cases, time may be critical and solutions must maximise the amount of visited mobiles and minimise the total needed time. This problem needs bi-criteria optimisation and is called a \acrlong{stdvrp}. To solve this problem, we implemented two route-building heuristics to generate feasible but not optimal solutions and a memory cache strategy was designed to speed-up the heuristics. Then, local search procedures were coded to look for improvements on the solutions and a Vertical Neighbourhood Descent was built to combine these procedures. After that, two metaheuristics based on a genetic algorithm with random keys and multi-start were used to generate better solutions on both time and amount criteria. The developed algorithms must be able to provide solutions quickly, so they were written using the C++ language and benchmarked with real-size problems. We present results of experiments on self-generated and well-known datasets. Results showed that the memory cache strategy divided the heuristic computation by ten. Several graphical representations for temporal and spatial dimensions were also provided along with a benchmark and Pareto frontiers charts to compare and rank the generated solutions. Efficiency and convergence charts justified the strategic decisions taken in the Vertical Neighbourhood Descent building process. The results obtained at the end of this project confirm that routing problems can be solved using constructive and iterative methods because of the large amount of possibilities on which an exact solver would fail or need an extremely long time to end. These results made it possible for us to validate a proof of concept in predicting how much time is needed to arrest drug smugglers on "go-fast" races.

\emph{Keywords: Decision support, Vehicle Routing Problem, Metaheuristic, Vertical Neighbourhood Descent, Bi-criteria optimisation.}
