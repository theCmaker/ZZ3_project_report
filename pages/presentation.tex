\chapter{Présentation du sujet}
	\section{Problème}
		Ce problème de tournées de véhicules correspond plus précisément à un STDVRP (\emph{Selective Time-Dependent Vehicle Routing Problem}), c'est-à-dire un problème où tous les clients ne sont pas nécessairement livrés (\emph{Selective}) et où les tournées ne peuvent excéder une certaine durée (Time-Dependent). Cette durée permet par ailleurs de modéliser une autonomie sur ces véhicules. Cela prend d'autant plus de sens si l'on souhaite appliquer ce problème sur une flotte de véhicules autonomes comme des drônes par exemple. L'objectif est de livrer un maximum de clients et de clore l'ensemble des tournées le plus tôt possible.

		Ce type de problème impose donc un espace géographique et un espace temporel dans lesquels les tournées sont réalisées.

		\subsection{Dimensions caractéristiques}
			Tout d'abord, les données sont astreintes à un cadre spatio-temporel qui peut être défini à l'aide des unités suivantes:
			\begin{itemize}
				\item Unité de temps (\emph{Time}, ou u.t.), pour modéliser l'autonomie des véhicules et le temps des tournées,
				\item Unité de distance (\emph{Distance}, ou u.d.), pour placer les véhicules et les clients,
				\item Unité de vitesse (\emph{Speed}, ou u.v.), correspondant à une unité de distance par unité de temps, pour calculer les déplacements.
			\end{itemize}
			Ces unités peuvent être combinées pour définir des coordonnées (\emph{Location}) et des déplacements (\emph{Direction}), dont nous nous servons fréquemment.

		\subsection{Vocabulaire}
			Nous avons également choisi d'employer certains termes plus en lien avec le sujet. Ainsi les véhicules sont nommés \emph{intercepteurs}, les clients sont nommés \emph{mobiles} et une livraison correspond à une \emph{interception}.

			Nous avons préféré choisir ces termes, car les mobiles sont en mouvement et les intercepteurs doivent changer de direction pour les rejoindre en un même endroit, à un même instant, aussi le terme d'interception fait plus de sens.

		\subsection{Propriétés et contraintes}
			Les calculs sont effectués dans le plan (et non dans l'espace). Les coordonnées de différentes entités sont à noter:
			\begin{itemize}
				\item \emph{mobiles}: position initiale (à l'instant $t=0$),
				\item \emph{dépôts}: position fixe,
				\item \emph{intercepteurs}: à $t=0$, la position d'un intercepteur est celle du dépôt auquel il est rattaché.
			\end{itemize}

			Ces trois entités définissent les données d'un problème. Elles sont décrites plus loin sous forme d'un ensemble de classes UML (figure~\ref{fig:problem-uml} page~\pageref{fig:problem-uml}).

			Chaque mobile et chaque intercepteur avance à une vitesse qui lui est propre. Le mobile doit suivre une direction donnée et l'intercepteur peut changer de direction après chaque interception.

			L'intercepteur doit rentrer à son dépôt dès la fin de sa tournée et son autonomie doit lui permettre ce retour. Enfin, la date de fin d'une tournée correspond à la date d'interception du dernier mobile de cette tournée.

		\subsection{Représentations}
			Dans les schémas et représentations spatiales que nous utiliserons pour décrire nos approches, nous emploierons les symboles suivants:
			\begin{itemize}
				\item Pour un mobile: le symbole (\tikz[baseline=-0.65ex]{\node[mobile,inner sep=0,outer sep=0]{\mobile};}) lorsqu'il est intercepté ou (\tikz[baseline=-0.65ex]{\node[mobile,uncaught,inner sep=0,outer sep=0]{\mobile};}) lorsqu'il ne l'est pas dans un schéma spatial et le symbole (\tikz[baseline=-0.65ex]{\node[mobile,draw]{};}) dans un schéma de tournée. Un vecteur vitesse (\tikz[baseline=-0.65ex]{\draw[speed] (0,0) -- (1,0);}) indique sa direction ainsi que sa vitesse. Sa trajectoire est tracée par une ligne pointillée (\tikz[baseline=-0.65ex]{\draw[direction] (0,0) -- (1,0);}).

				\item Pour un dépôt, un symbole (\tikz[baseline=-0.65ex]{\node[interceptor,inner sep=0,outer sep=0]{\interceptor};}) indique sa position.

				\item Pour un intercepteur : son tracé est réalisé en traits pleins et part d'un dépôt. Afin d'en faciliter la lecture, le retour au dépôt n'est pas représenté. Le tracé est ponctué de symboles (\tikz[baseline=-0.65ex]{\node[route,inner sep=0,outer sep=0]{\mobile};}) aux emplacements où l'interception d'un mobile a eu lieu. La date de fin d'une tournée est indiquée à proximité du dernier mobile intercepté.
			\end{itemize}
			Les figures~\ref{subfig:spatial} et~\ref{subfig:schema} présentent un schéma spatial et un schéma de tournée.

			\begin{figure}[h!]
				\begin{subfigure}[b]{.45\linewidth}
					\centering
					\begin{tikzpicture}[scale=.5]
						\draw[grided,step=1.0,thin] (0,0) grid (12,12);
\draw[color=Gray] (0,0) -- coordinate (x axis mid) (12,0);
\draw[color=Gray] (0,0) -- coordinate (x axis mid) (0,12);
\foreach \x in {0,...,12}
\draw[color=Gray] (\x,1pt) -- (\x,-3pt) node[anchor=north] {\x};
\foreach \y in {0,...,12}
\draw[color=Gray] (1pt,\y) -- (-3pt,\y) node[anchor=east] {\y};
\node[interceptor] (D0) at (12,12) {\interceptor};
\node (I0) at (D0) {};
\node[interceptor] at ($ (I0) + (315:0.4) $) {$I_0$};
\node[mobile,anchor=center] (M0) at (2,5) {\mobile};
\node[mobile] at (M0.south east) {$M_{0}$};
\draw[speed] (M0.center) -- ($ (M0.center) + (1,0.3) $);
\node[mobile,anchor=center] (M1) at (4,1) {\mobile};
\node[mobile] at (M1.south east) {$M_{1}$};
\draw[speed] (M1.center) -- ($ (M1.center) + (0.5,0.5) $);
\node[mobile,uncaught,anchor=center] (M2) at (6,10) {\mobile};
\node[mobile] at (M2.south east) {$M_{2}$};
\draw[speed,uncaught] (M2.center) -- ($ (M2.center) + (-4,0) $);
\draw[direction] (M0.center) -- (6.11793,6.23538);
\draw[route] (12,12) -- (6.11793,6.23538);
\node[route] at (6.11793,6.23538) {\mobile};
\draw[direction] (M1.center) -- (6.70783,3.70783);
\draw[route] (6.11793,6.23538) -- (6.70783,3.70783);
\node[route] at (6.70783,3.70783) {\mobile};
\draw[route](6.70783,3.70783) node[anchor=north east] {$t_{0}=5.41567$};
					\end{tikzpicture}
					\subcaption{Représentation spatiale}
					\label{subfig:spatial}
				\end{subfigure}
				\hfill
				\begin{subfigure}[b]{.45\linewidth}
					\centering
					\vfill
					\begin{tikzpicture}[schema]
						\begin{scope}[start chain=trunk]
	\node[interceptor, on chain, label=left:$I_0$] {};
	\node[mobile, on chain, join, label=above:$M_0$] {};
	\node[mobile, on chain, join, label=above:$M_1$] {};
	\node[interceptor, on chain, join] {};
\end{scope}
					\end{tikzpicture}
					\vfill
					\null
					\subcaption{Représentation schématique d'une tournée}
					\label{subfig:schema}
				\end{subfigure}

				\begin{subfigure}{\linewidth}
					\centering
					\begin{tikzpicture}[yscale=.5]
						\draw[grided,step=1.0,thin] (0,0) grid (13,10);
\draw[color=Gray] (0,0) -- coordinate (x axis mid) (13,0);
\draw[color=Gray] (0,0) -- coordinate (x axis mid) (0,10);
\foreach \x in {0,...,13}
\draw[color=Gray] (\x,1pt) -- (\x,-3pt) node[anchor=north] {\x};
\foreach \y in {0,...,10}
\draw[color=Gray] (1pt,\y) -- (-3pt,\y) node[anchor=east] {\y};
\draw[thick,color=Green](0,0) -| (1.51755,1) -| (1.73132,2) -| (2.6224,3) -| (3.31184,4) -| (5.15684,5) node[pos=1.0,mark size=5pt] {\pgfuseplotmark{star}};
\draw[dashed,thick,color=Green](5.15684,4) -- (5.15684,0) node[anchor=north west, rotate=90] {$t_{1}=5.16$};
\draw[thick,color=OrangeRed](0,0) -| (0.727136,1) -| (1.43753,2) -| (2.10416,3) -| (2.60203,4) -| (3.35262,5) -| (4.21061,6) -| (4.96927,7) -| (6.05742,8) -| (7.98504,9) -| (12.5556,10) node[pos=1.0,mark size=5pt] {\pgfuseplotmark{star}};
\draw[dashed,thick,color=OrangeRed](12.5556,9) -- (12.5556,0) node[anchor=north west, rotate=90] {$t_{2}=12.56$};
\draw[thick,color=Cyan](0,0) -| (2.37456,1) -| (4.72451,2) node[pos=1.0,mark size=5pt] {\pgfuseplotmark{star}};
\draw[dashed,thick,color=Cyan](4.72451,1) -- (4.72451,0) node[anchor=north west, rotate=90] {$t_{3}=4.72$};
\draw[thick,color=Purple](0,0) -| (0.558335,1) -| (1.82611,2) -| (3.20879,3) -| (7.98206,4) node[pos=1.0,mark size=5pt] {\pgfuseplotmark{star}};
\draw[dashed,thick,color=Purple](7.98206,3) -- (7.98206,0) node[anchor=north west, rotate=90] {$t_{4}=7.98$};

					\end{tikzpicture}
					\subcaption{Exemple de schéma temporel à plusieurs tournées}
					\label{subfig:example-temporal}
				\end{subfigure}
				\caption{Exemples de schémas}
				\label{fig:example}
			\end{figure}

			Nous avons également entrevu la nécessité de proposer une représentation temporelle pour décrire les tournées. En effet, ce type de représentation permet d'observer la répartition de la charge sur les tournées. La notion de distance disparaît et avec elle la notion de vitesse. On peut ainsi lire plus facilement des indicateurs comme la durée nécessaire pour réaliser une partie d'une tournée ou pour atteindre le $n$\ieme{} client.

			Dans ce cas, la représentation prend la forme d'une frise chronologique où l'on fait correspondre à chaque tournée une courbe en escaliers franchissant un palier à chaque date correspondant à une interception. Ainsi, on lit le temps en abscisse et le nombre de mobiles interceptés en ordonnée. Pour repérer plus facilement la fin d'une tournée, un marqueur est apposé et sa position est projetée sur l'axe des abscisses pour faciliter la lecture de la date. Un exemple est proposé sur la figure~\ref{subfig:example-temporal}.

		

	\section{Etat de l'art}
		Comme pour tous les problèmes de tournées de véhicules (ou \acrlong{vrp}), il s'agit d'un problème NP-Difficile et il n'est donc pas possible d'obtenir une solution optimale dans un temps raisonnable par la force brute, c'est-à-dire en calculant toutes les combinaisons existantes. Il est alors nécessaire de mettre en place des méthodes d'approximation itératives pour déterminer la meilleure solution.

		Ce problème de \acrlong{stdvrp} avait déjà été abordé dans un projet réalisé en première année et celui-ci permettait de travailler avec un seul intercepteur pour obtenir des résultats comme celui que l'on peut voir sur la figure \ref{subfig:spatial}. Le programme était écrit en langage C et prenait en entrée un fichier décrivant les données du problème. Le listing \ref{lst:old_input_file} correspond au fichier d'entrée menant au résultat de la figure \ref{subfig:spatial}.

		Ce fichier avait été conçu pour en faciliter l'évolution. Bien que le programme ne puisse travailler qu'avec un seul intercepteur, il était possible d'en renseigner plusieurs, ainsi que plusieurs dépôts. Nous avons donc conservé un format de fichier similaire pour notre projet, il sera présenté plus tard.

		\begin{code}
			\textfile{files/old_input_file.txt}
			\captionof{listing}{Fichier d'entrée original}
			\label{lst:old_input_file}
		\end{code}

	\section{Objectifs}
		Notre projet avait pour but de réutiliser le travail initial en généralisant les heuristiques de construction pour qu'elles puissent fonctionner avec de multiples intercepteurs, ainsi que d'optimiser les résultats selon deux critères : le nombre de mobiles visités et le temps requis pour l'interception.
		Plusieurs étapes étaient nécessaires pour la réalisation ce projet.

		Tout d'abord, un travail de révision et d'optimisation était requis avant de pouvoir poursuivre ce qui avait été fait. Le code initial ayant été réalisé en langage C, il devait être traduit en C++ dans une architecture adaptée et orientée objet. Puis le code obtenu a été modifé pour généraliser le problème non plus à un seul, mais à plusieurs intercepteurs pouvant provenir de différents dépôts. Une optimisation a été apportée dans les fonctions de calcul d'interception et les heuritisques de construction. Il était aussi nécessaire de revoir les schémas, notamment en ajoutant une représentation temporelle en plus de la représentation spatiale déjà existante, que l'on a adaptée au contexte.

		Ensuite, les objectifs d'amélioration des solutions ont été définis tout au long du projet, en suivant les étapes qui ont été réalisées. La mise en place d'une recherche locale a permis de générer les premières améliorations d'une solution avec la mise en place d'opérations d'insertion et d'extraction de mobiles. Ces mouvements simples ont ensuite été combinés pour apporter à la recherche locale de nombreuses possibilités afin d'optimiser la solution courante. Ces mouvements ont finalement été appliqués dans le cadre d'une métaheuristique.
		
		Il est aussi important de tenir compte de paramètres supplémentaires pour apporter plus de réalisme tels que l'autonomie d'un intercepteur et l'aspect stochastique sur la date d'interception. Ce dernier point permettrait la modélisation d'un délai supplémentaire pour valider l'interception: la durée nécessaire à une identification formelle.
		
		
		Plusieurs mesures ont été fixées afin de faciliter les calculs. Les déplacements des mobiles et des intercepteurs se font systématiquement dans le plan (l'altitude n'aura donc pas d'influence). Les trajectoires des mobiles doivent aussi être rectilignes et leur vitesse constante. Il n'y a pas de variation de vitesse ni de direction pour les intercepteurs entre deux interceptions consécutives.

		\begin{figure}[h!]
			\centering
			\begin{tikzpicture}
				\umlclass[anchor=center]{Problem}{
}{
	+ Problem(string)\\
	+ Problem(Problem)\\
	+ nbMobiles() : int\\
	+ nbInterceptors() : int\\
	+ nbDepots() : int\\
	+ mobiles() : vector<Mobile>\\
	+ interceptors() : vector<Interceptor>\\
	+ depots() : vector<Depot>\\
	+ write(string) : void\\
}

\umlclass[x=8.5,y=-4.5,anchor=north]{Interceptor}{
	-- id : int\\
	-- position : Location\\
	-- speed : Speed\\
	-- range : Time\\
	\umlstatic{+ INTERCEPTION\_TIME\_NO\_FUEL : Time}
}{
	+ Interceptor(Location, Speed, int, Time)\\
	+ id() : int\\
	+ position() : Location\\
	+ speed() : Speed\\
	+ range() : Time\\
	+ depot() : Depot\\
	+ position(Location) : void\\
	+ speed(Speed) : void\\
	+ range(Time) : void\\
	+ computeInterception(Location, Mobile, Time) : Time\\
	+ timeFromTo(Location, Location) : Time
}

\umlclass[y=-4.5,anchor=north]{Mobile}{
	-- id : int\\
	-- position : Location\\
	-- direction : Direction\\
}{
	+ Mobile(Location, Direction, int)\\
	+ id() : int\\
	+ position() : Location\\
	+ position(Time) : Location\\
	+ direction() : Direction\\
	+ speed() : Speed\\
	+ position(Location) : void\\
	+ direction(Direction) : void
}

\umlclass[x=8.5,anchor=center]{Depot}{
	-- id : int\\
	-- position : Location\\
}{
	+ Depot(Location, Direction, int)\\
	+ id() : int\\
	+ position() : Location\\
	+ interceptors() : vector<Interceptor>\\
	+ position(Location) : void\\
	+ addInterceptor(Interceptor) : void
}

\umlunicompo[geometry=--,arg=depots, mult1=1, mult2=*]{Problem}{Depot}
\umlunicompo[geometry=--,arg=mobiles, mult1=1, mult2=*]{Problem}{Mobile}
\umlunicompo[geometry=--,arg=interceptors, mult1=1, mult2=*]{Problem}{Interceptor}
\umluniaggreg[geometry=--,arg=interceptors, mult1=1, mult2=*,anchor1=-70, anchor2=70]{Depot}{Interceptor}
\umluniaggreg[geometry=--,arg=depot, mult1=1, mult2=1,anchor1=110, anchor2=-110]{Interceptor}{Depot}
			\end{tikzpicture}
			\caption{Classes définissant un problème}
			\label{fig:problem-uml}
		\end{figure}

		Pour assurer un suivi régulier du projet, nous avons planifié des réunions avec M. Duhamel chaque semaine dans la mesure du possible ou toutes les deux semaines.

		Ainsi nous avons construit le diagramme de Gantt prévisionnel présenté sur la figure~\ref{fig:gantt_previsionnel}. Il est suivi par le diagramme de Gantt réel (figure~\ref{fig:gantt_reel}).

		\begin{figure}[h!]
	\shorthandoff{:!}
	\centering
	\begin{ganttchart}%
		[y unit title=0.7cm,
		x unit=0.05cm,
		hgrid=true,
		vgrid={*6{white},*1{gray,dotted}},%
		title/.style={draw,fill=Red!30!white},
		title label font=\scriptsize,
		bar label font=\color{black!50},
		group right shift=0,
		group left shift=0,
		group peaks width=2,
		group peaks tip position=0,
		group peaks height=.3,
		bar left shift=0.01,
		bar right shift=-.01,
		bar incomplete/.append style={fill=Red},
		progress label text={},
		time slot format={isodate-yearmonth}]{2016-10}{2017-04}
		%En-têtes
		\gantttitlecalendar{year, month=shortname}\\
		%Contenu
		\ganttgroup{Lancement}{2016-10-20}{2016-12-05}\\
		\ganttbar[progress=0,name=prisemain]{Prise en main du code existant}{2016-10-20}{2016-11-03} \\
		\ganttbar[progress=0,name=cppconv]{Conversion en C++}{2016-10-20}{2016-11-09} \\
		\ganttbar[progress=0,name=multii]{Adaptation multi-intercepteurs}{2016-11-03}{2016-11-23}\\
		\ganttbar[progress=0,name=graph]{Rendu graphique}{2016-11-03}{2016-12-05} \\[grid]
		%
		\ganttgroup{Optimisation, robustesse}{2016-11-01}{2017-02-22}\\
		\ganttbar[progress=0,name=tests]{Tests}{2016-11-01}{2017-02-22}\\
		\ganttbar[progress=0,name=optim]{Optimisation de l'insertion}{2016-11-15}{2016-12-15}\\
		\ganttbar[progress=0,name=doc]{Documentation}{2016-12-19}{2017-02-22}\\
		%
		\ganttgroup{Méta-Heuristiques}{2016-12-15}{2017-02-15}\\
		\ganttbar[progress=0,name=rl]{Recherche locale}{2016-12-15}{2017-01-16}\\
		\ganttbar[progress=0,name=policies]{Politiques de mouvement}{2017-01-01}{2017-01-25}\\
		\ganttbar[progress=0,name=bic]{Bi-critères}{2017-01-20}{2017-02-10}\\
		\ganttbar[progress=0,name=mh]{VND + BRKGA}{2017-01-01}{2017-02-15}\\
		%
		\ganttgroup{Aspect stochastique}{2017-02-15}{2017-03-05}\\
		\ganttbar[progress=100,name=stochas]{Probabilité tps interception}{2017-02-15}{2017-02-25}\\
		%
		\ganttgroup{Résultats}{2017-02-15}{2017-03-05}\\
		\ganttbar[progress=0,name=results]{Résultats}{2017-02-15}{2017-03-05}
		%
	\end{ganttchart}
	\caption{Diagramme de Gantt prévisionnel}
	\label{fig:gantt_previsionnel}
	\shorthandon{:!}
\end{figure}

		\begin{figure}[h!]
	\shorthandoff{:!}
	\centering
	\begin{ganttchart}%
		[y unit title=0.7cm,
		x unit=0.05cm,
		hgrid=true,
		vgrid={*6{white},*1{gray,dotted}},%
		title/.style={draw,fill=Red!30!white},
		title label font=\scriptsize,
		bar label font=\color{black!50},
		group right shift=0,
		group left shift=0,
		group peaks width=2,
		group peaks tip position=0,
		group peaks height=.3,
		bar left shift=0.01,
		bar right shift=-.01,
		bar incomplete/.append style={fill=Red},
		progress label text={},
		time slot format={isodate-yearmonth}]{2016-10}{2017-04}
		%En-têtes
		\gantttitlecalendar{year, month=shortname}\\
		%Contenu
		\ganttgroup{Lancement}{2016-10-20}{2016-12-05}\\
		\ganttbar[progress=0,name=prisemain]{Prise en main du code existant}{2016-10-20}{2016-11-03} \\
		\ganttbar[progress=0,name=cppconv]{Conversion en C++}{2016-10-20}{2016-11-09} \\
		\ganttbar[progress=0,name=multii]{Adaptation multi-intercepteurs}{2016-11-03}{2016-11-23}\\
		\ganttbar[progress=0,name=graph]{Rendu graphique}{2016-11-03}{2016-12-05} \\[grid]
		%
		\ganttgroup{Optimisation, robustesse}{2016-11-01}{2017-02-22}\\
		\ganttbar[progress=0,name=tests]{Tests unitaires}{2016-11-01}{2017-02-22}\\
		\ganttbar[progress=0,name=optim]{Optimisation de l'insertion}{2016-12-03}{2017-01-09}\\
		\ganttbar[progress=0,name=doc]{Documentation}{2016-12-19}{2017-02-22}\\
		%
		\ganttgroup{Méta-Heuristiques}{2017-01-12}{2017-03-05}\\
		\ganttbar[progress=0,name=rl]{Recherche locale}{2017-01-12}{2017-02-16}\\
		\ganttbar[progress=0,name=policies]{Politiques de mouvement}{2017-01-29}{2017-02-15}\\
		\ganttbar[progress=0,name=bic]{Bi-critères}{2017-02-08}{2017-02-23}\\
		\ganttbar[progress=0,name=mh]{VND + BRKGA}{2017-01-21}{2017-03-05}\\
		%
		\ganttgroup{Résultats}{2017-02-27}{2017-03-05}\\
		\ganttbar[progress=0,name=results]{Résultats}{2017-02-27}{2017-03-05}\\
		%
		\ganttgroup{Innatendu}{2017-02-08}{2017-02-15}\\
		\ganttbar[progress=0,name=results]{Changement des conditions du problème}{2017-02-08}{2017-02-15}
		%
		%Liens avec saut
	\end{ganttchart}
	\caption{Diagramme de Gantt réel}
	\label{fig:gantt_reel}
	\shorthandon{:!}
\end{figure}
		
		\clearpage
		Dès le début du projet, nous avons pu déterminer les principales étapes à réaliser telles que la conversion et l'optimisation du code, l'implémentation d'une recherche locale et de métaheuristiques ainsi que la nécessité de tester régulièrement notre travail. Cela nous a permis d'avoir une bonne vision du projet dans son ensemble et de répartir les tâches sur les mois dédiés à la réalisation de celui-ci. Avec le diagramme de Gantt réel, on peut constater que le temps estimé pour les tâches était plutôt cohérent et que presque la totalité d'entre elles ont pu être achevées. Seul l'ajout d'une incertitude sur le temps d'interception (qui ne serait plus une date, mais une probabilité) n'a pas pu être réalisé.
		
		Quelques ralentissements ont affecté l'avancement du projet, notamment au mois de décembre et au cours du développement des mouvements pour la recherche locale dont l'implémentation s'est avérée plus complexe que prévue. Ces retards ont cependant pu être rattrapés pendant les mois de janvier et de février grâce à un travail un peu plus condensé sur chaque semaine.
		
		Nous avons aussi dû légèrement réviser la modélisation du problème pour le rendre plus réaliste en ajoutant une autonomie à chaque intercepteur. En effet, nous ne considérions à l'origine aucun retour au dépôt après la dernière interception. Désormais, la validité d'une interception dépend aussi de la capacité de l'intercepteur à revenir à son point de départ, même si ce temps de retour n'est pas compté dans le temps total d'une tournée. La bonne conception des classes et des méthodes a permis de rajouter rapidement et sans grosses modifications ce changement dans les règles du problème.
		
		L'obtention des résultats a aussi été un peu plus longue que prévue, car elle a permis de mettre en évidence quelques bugs et erreurs de programmation, notamment dans la validation d'un mouvement. La correction de ces erreurs nous a permis néanmoins d'assurer des résultats plus fiables.
