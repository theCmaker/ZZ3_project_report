\chapter{Présentation du sujet}
    \section{Problème}
    	Ce problème de tournées de véhicules correspond plus précisément à un STDVRP (\emph{Selective Time-Dependent Vehicle Routing Problem}), c'est-à-dire un problème où tous les clients ne sont pas nécessairement livrés (\emph{Selective}) et où les tournées ne peuvent excéder une certaine durée (Time-Dependent). Cette durée permet par ailleurs de modéliser une autonomie sur ces véhicules. Cela prend d'autant plus de sens si l'on souhaite appliquer ce problème sur une flotte de véhicules autonomes comme des drônes par exemple.

    	Afin de fixer certains paramètres nous avons déterminé la nécessité d'utiliser les unités suivantes:
    	\begin{itemize}
    		\item Unité de temps (u.t.), pour modéliser l'autonomie des véhicules et le temps des tournées,
    		\item Unité de distance (u.d.), pour placer les véhicules et les clients,
    		\item Unité de vitesse (u.v.), correspondant à une unité de distance par unité de temps, pour calculer les déplacements.
    	\end{itemize}

    	Nous avons également choisi d'employer certains termes plus en lien avec le sujet. Ainsi les véhicules sont nommés \emph{intercepteurs}, les clients sont nommés \emph{mobiles} et une livraison correspond à une \emph{interception}. Nous avons préféré choisir ces termes car les mobiles sont en mouvement et les intercepteurs doivent changer de direction pour les rejoindre en un même endroit à un même instant, aussi le terme d'interception fait plus de sens.

    	Les calculs sont effectués dans le plan (et non dans l'espace). Les coordonnées de différentes entités sont à noter:
    	\begin{itemize}
    		\item \emph{mobiles}: position initiale (à l'instant $t=0$),
    		\item \emph{dépôts}: position fixe
    		\item \emph{intercepteurs}: à $t=0$, la position d'un intercepteur est celle du dépôt auquel il est rattaché.
    	\end{itemize}

    	Chaque mobile et chaque intercepteur avance à une vitesse qui lui est propre. Le mobile doit suivre une direction donnée, et l'intercepteur peut changer de direction après chaque interception.

    	L'intercepteur doit rentrer à son dépôt dès la fin de sa tournée, et son autonomie doit lui permettre ce retour. Enfin, la date de fin d'une tournée correspond à la date d'interception du dernier mobile de cette tournée.

    	Dans les schémas que nous utiliserons pour décrire nos approches, nous emploierons les symboles suivants:
    	\begin{itemize}
    		\item Pour un mobile: le symbole (\tikz[baseline=-0.5ex]{\node[mobile,caught,inner sep=0,outer sep=0]{\mobile};}) lorsqu'il est intercepté, ou (\tikz[baseline=-0.5ex]{\node[mobile,inner sep=0,outer sep=0]{\mobile};}) lorsqu'il ne l'est pas dans un schéma spatial, et le symbole $\oplus$ dans un schéma de tournée. Un vecteur vitesse (\tikz[baseline=-0.5ex]{\draw[speed] (0,0) -- (1,0);}) indique sa direction ainsi que sa vitesse. Sa trajectoire est tracée par une ligne pointillée (\tikz[baseline=-0.5ex]{\draw[route] (0,0) -- (1,0);}).
    		
    		\item Pour un dépôt, un symbole (\tikz[baseline=-0.5ex]{\node[interceptor,inner sep=0,outer sep=0]{\interceptor};}) indique sa position.
    		
    		\item Pour un intercepteur: son tracé est réalisé en traits pleins, et part d'un dépôt. Afin d'en faciliter la lecture, le retour au dépôt n'est pas représenté. Le tracé est ponctué de symboles (\tikz[baseline=-0.5ex]{\node[interceptor,inner sep=0,outer sep=0]{\mobile};}) aux emplacements où l'interception d'un mobile a eu lieu. La date de fin d'une tournée est indiquée à proximité du dernier mobile intercepté.
    	\end{itemize}
    \section{Etat de l'art}
    \section{Travail à réaliser}