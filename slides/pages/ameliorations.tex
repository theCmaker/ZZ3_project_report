\subsection{Améliorations}
\begin{frame}
	\frametitle{Recherche Locale}
	\begin{block}{Principe}
		\begin{itemize}
			\item Effectuer des mutations
			\item Améliorer les tournées
			\item Conserver la réalisabilité
			\item Travaille sur une seule solution
		\end{itemize}
	\end{block}
	\vspace*{1em}
	\begin{block}{Mouvements}
		\begin{itemize}
			\item Définition de mouvements simples dans les tournées
			\item Huit mouvements implémentés
		\end{itemize}
	\end{block}
	
\end{frame}
\begin{frame}
	\frametitle{Mouvements améliorants (1)}
	\framesubtitle{Insertion/Extraction}
	\begin{columns}
		\begin{column}{.53\linewidth}
			\begin{block}{Insertion}
				\centering
				\begin{tikzpicture}[scale=.7,schema,transform shape,thick]
					\begin{scope}[start chain,x=0,y=0]
	\node[interceptor, on chain] {};
	\node[mobile, on chain, join] {};
	\node[mobile, on chain, join] (B) {};
	\node[mobile, on chain, join] (A) {};
	\node[mobile, on chain, join] {};
	\node[interceptor, on chain, join] {};
\end{scope}
\node[right= .5 of B,cross] (X) {$\times$};
\node[mobile, below= .5 of X,red] (I) {};
\draw[correction,dotted] (B) -- (I);
\draw[correction,dotted] (I) -- (A);

				\end{tikzpicture}
			\end{block}
			\begin{itemize}
				\item Mobile non intercepté
				\item Améliore le nombre de mobiles
				\item Ne doit pas détériorer le temps
			\end{itemize}
		\end{column}
		\begin{column}{.53\linewidth}
			\begin{block}{Extraction}
				\centering
				\begin{tikzpicture}[scale=.7,schema,transform shape,thick]
					\begin{scope}[start chain, x=0,y=0]
	\node[interceptor, on chain] {};
	\node[mobile, on chain, join] {};
	\node[mobile, on chain, join] (B) {};
	\node[mobile, on chain, join,red] (E) {};
	\node[mobile, on chain, join] (A) {};
	\node[interceptor, on chain, join] {};
\end{scope}
\draw[correction,dotted] (B) to[bend right] (A);
\draw[correction] (E) -- ($ (E) + (0,1) $);
\node[right= .5 of B,cross] {$\times$};
\node[right= .5 of E,cross] {$\times$};

				\end{tikzpicture}
			\end{block}
			\begin{itemize}
				\item Sur la pire tournée
				\item Réduit le temps max
				\item Nécessite un bon pilotage
			\end{itemize}
		\end{column}
	\end{columns}
	\vspace*{2em}
	\textbf{Mouvement de remplacement} : suppression puis insertion
\end{frame}
\begin{frame}
	\frametitle{Mouvements améliorants (2)}
	\framesubtitle{Déplacement}
	\begin{block}{Déplacement}
		\centering
		\begin{tikzpicture}[scale=.7,schema,transform shape,thick]
			\begin{scope}[start chain]
	\node[interceptor, on chain] {};
	\node[mobile, on chain, join] (B) {};
	\node[mobile, on chain, join,red] (M) {};
	\node[mobile, on chain, join] (A) {};
	\node[interceptor, on chain, join] {};
\end{scope}
\begin{scope}[start chain, yshift=-1.25cm]
	\node[interceptor, on chain] {};
	\node[mobile, on chain, join] {};
	\node[mobile, on chain, join] (I) {};
	\node[mobile, on chain, join] (IA) {};
	\node[mobile, on chain, join] {};
	\node[interceptor, on chain, join] {};
\end{scope}
\draw[correction,dotted] (I) -- node[stepnode] {1} (M);
\draw[correction,dotted] (M) -- node[stepnode] {2} (IA);
\draw[correction,dotted] (B) to[bend left] node[stepnode] {3} (A);
\node[right= .5 of M,cross] {$\times$};
\node[right= .5 of B,cross] {$\times$};
\node[right= .5 of I,cross] (X) {$\times$};
\draw[correction] (M) -- (X);


		\end{tikzpicture}
	\end{block}
	Deux cas :
	\begin{itemize}
		\item Déplacement d'un mobile dans sa tournée
		\item Déplacement dans une autre tournée
	\end{itemize}
	\vspace*{1em}
	\begin{itemize}
		\item Meilleure organisation des tournées
		\item Gain en temps
	\end{itemize}
	\vspace*{.5em}
	\textbf{Mouvement swap} : échange de deux mobiles
	
\end{frame}
\begin{frame}
	\frametitle{Mouvements améliorants (3)}
	\framesubtitle{2-Opt}
	\begin{block}{2-Opt}
		\centering
		\begin{tikzpicture}[scale=.7,schema,transform shape,thick]
			\begin{scope}[start chain]
	\node[interceptor, on chain] {};
	\node[mobile, on chain, join] (B1) {};
	\node[mobile, on chain, join] (S1) {};
	\node[mobile, on chain, join] (A1) {};
	\node[mobile, on chain, join] (L1) {};
	\node[interceptor, on chain, join] (E1) {};
\end{scope}
\begin{scope}[start chain,yshift=-1.25cm]
	\node[interceptor, on chain] {};
	\node[mobile, on chain, join] (B2) {};
	\node[mobile, on chain, join] (S2) {};
	\node[mobile, on chain, join] (A2) {};
	\node[mobile, on chain, join] (L2) {};
	\node[interceptor, on chain, join] (E2) {};
\end{scope}
\draw[correction,rounded corners] ($(A1) + (-.4,.3)$) rectangle ($(L1) + (.3,-.3) $);
\draw[correction,rounded corners] ($(A2) + (-.4,.3)$) rectangle ($(L2) + (.3,-.3) $);
\draw[correction,dotted] (S1) -- node[stepnode,near start] {1} (A2);
\draw[correction,dotted] (S2) -- node[stepnode,near start] {3} (A1);
\draw[correction,dotted] (L1) -- node[stepnode,near start] {4} (E2);
\draw[correction,dotted] (L2) -- node[stepnode,near start] {2} (E1);
\node[right= .5 of S1,cross] {$\times$};
\node[right= .5 of S2,cross] {$\times$};
\node[right= .5 of L1,cross] {$\times$};
\node[right= .5 of L2,cross] {$\times$};

\draw[correction,<->] ($(L1) + (-.75,-.32) $) -- ($(L2) + (-.75,.32) $);

		\end{tikzpicture}
	\end{block}
	\begin{itemize}
		\item Deux tournées
		\item Inversion des séquences d'un mobile de la tournée jusqu'à la fin
		\item Gain en temps
	\end{itemize}
\end{frame}
\begin{frame}
	\frametitle{Optimisation bi-critères}
	\begin{columns}
		\begin{column}{.37\linewidth}
			\begin{alertblock}{Objectifs}
				\begin{itemize}
					\item Nombre de mobiles
					\item Temps
				\end{itemize}
			\end{alertblock}
			\begin{exampleblock}{Améliorants}
				\begin{itemize}
					\item Premier
					\item Meilleur
					\item[$\Rightarrow$] Politiques
				\end{itemize}
			\end{exampleblock}
		\end{column}
		\begin{column}{.63\linewidth}
			\begin{center}
			\begin{tikzpicture}[xscale=1,yscale=0.8]
  \clip (-1,-1) rectangle (7,7);
  
  \definecolor{pLineColor}{RGB}{128,0,0}
  \definecolor{pPointColor}{RGB}{0,40,240}

  
  \only<5>{
    \fill[thick,dashed,color=Blue!20] (4.4,5.5) -- (4.4,0) -- (4,0) -- (4,5) -- (0,5) -- (0,5.5) --cycle;
  }
  
  \draw[thick,->,>=latex] (0,0) --node[pos=1,above]{\textit{Temps}} (0,6);
  \draw[thick,->,>=latex] (0,0) --node[pos=1,below left]{\textit{\# ratés}} (6,0);

  \only<2->{
    \draw[dotted,thick] (4,5) --node[pos=1,left] {$y$} (0,5) ;
    \draw[dotted,thick] (4,5) --node[pos=1,below] {$x$} (4,0);
    \node[draw,color=pPointColor,fill=pPointColor, inner sep = 0pt, minimum size=2mm] at(4,5) {};
  }

  \only<5>{
    \draw[thick,dashed] (4.4,6) --node[pos=0,above] {$110\%\: x$} (4.4,0);
    \draw[thick,dashed] (4.9,5.5) --node[pos=0,right] {$110\%\: y$} (0,5.5);
  }


  \only<3-> {
    \node[draw,color=pPointColor,fill=pPointColor, inner sep = 0pt, minimum size=2mm] at(0.8, 2.866) {};
    \draw[thick,color=GoodGreen,snake=snake] (4, 5) -- (1, 3);
    \draw[thick,->,>=latex,color=GoodGreen] (1, 3) -- (0.8, 2.866);
  }
  \only<4-> {
    \draw[thick,->,>=latex,color=orange,dashed] plot [smooth] coordinates {(4,5) (1.6,5.25) (0.8,2.866)} ;
  }


\end{tikzpicture}

			\end{center}
		\end{column}
	\end{columns}
\end{frame}

\subsection{Mise en \oe uvre}
\begin{frame}
	\frametitle{Vertical Neighbourhood Descent}
	\framesubtitle{Concept}
	\begin{block}{Fonctionnement}
		\begin{itemize}
			\item Automatisation des mouvements dans un algorithme
			\item Pas d'amélioration => passer au mouvement suivant
			\item Amélioration => revenir au premier mouvement
		\end{itemize}
	\end{block}
	\vspace*{1em}
	\begin{block}{Mouvements}
		\begin{itemize}
			\item L'ordre des mouvements est crucial
			\item Détermine l'effacité de la VND
		\end{itemize}
	\end{block}
\end{frame}
\begin{frame}
	\frametitle{Vertical Neighbourhood Descent}
	\framesubtitle{Calibrage}
	\begin{alertblock}{Objectifs}
		\begin{itemize}
			\item Convergence
			\item Stabilité
			\item Efficacité
		\end{itemize}
	\end{alertblock}
	\begin{block}{Moyens}
		\begin{itemize}
			\item 8 mouvements
			\item 2 politiques
		\end{itemize}
	\end{block}
	\begin{exampleblock}{Solutions}
		\begin{itemize}
			\item 10 millions!
			\item Combinaisons aléatoires
			\item Instances aléatoires
		\end{itemize}
	\end{exampleblock}
\end{frame}
\begin{frame}
	\frametitle{Vertical Neighbourhood Descent}
	\framesubtitle{Résultats}
	\TODO{CONTENU}
	+ Comparaison heuristiques insertion
\end{frame}
